\documentclass[]{book}
\usepackage{lmodern}
\usepackage{amssymb,amsmath}
\usepackage{ifxetex,ifluatex}
\usepackage{fixltx2e} % provides \textsubscript
\ifnum 0\ifxetex 1\fi\ifluatex 1\fi=0 % if pdftex
  \usepackage[T1]{fontenc}
  \usepackage[utf8]{inputenc}
\else % if luatex or xelatex
  \ifxetex
    \usepackage{mathspec}
  \else
    \usepackage{fontspec}
  \fi
  \defaultfontfeatures{Ligatures=TeX,Scale=MatchLowercase}
\fi
% use upquote if available, for straight quotes in verbatim environments
\IfFileExists{upquote.sty}{\usepackage{upquote}}{}
% use microtype if available
\IfFileExists{microtype.sty}{%
\usepackage{microtype}
\UseMicrotypeSet[protrusion]{basicmath} % disable protrusion for tt fonts
}{}
\usepackage[margin=1in]{geometry}
\usepackage{hyperref}
\hypersetup{unicode=true,
            pdftitle={Lewis \& Clark BLT cluster},
            pdfauthor={Watzek DI, etc.},
            pdfborder={0 0 0},
            breaklinks=true}
\urlstyle{same}  % don't use monospace font for urls
\usepackage{natbib}
\bibliographystyle{apalike}
\usepackage{longtable,booktabs}
\usepackage{graphicx,grffile}
\makeatletter
\def\maxwidth{\ifdim\Gin@nat@width>\linewidth\linewidth\else\Gin@nat@width\fi}
\def\maxheight{\ifdim\Gin@nat@height>\textheight\textheight\else\Gin@nat@height\fi}
\makeatother
% Scale images if necessary, so that they will not overflow the page
% margins by default, and it is still possible to overwrite the defaults
% using explicit options in \includegraphics[width, height, ...]{}
\setkeys{Gin}{width=\maxwidth,height=\maxheight,keepaspectratio}
\IfFileExists{parskip.sty}{%
\usepackage{parskip}
}{% else
\setlength{\parindent}{0pt}
\setlength{\parskip}{6pt plus 2pt minus 1pt}
}
\setlength{\emergencystretch}{3em}  % prevent overfull lines
\providecommand{\tightlist}{%
  \setlength{\itemsep}{0pt}\setlength{\parskip}{0pt}}
\setcounter{secnumdepth}{5}
% Redefines (sub)paragraphs to behave more like sections
\ifx\paragraph\undefined\else
\let\oldparagraph\paragraph
\renewcommand{\paragraph}[1]{\oldparagraph{#1}\mbox{}}
\fi
\ifx\subparagraph\undefined\else
\let\oldsubparagraph\subparagraph
\renewcommand{\subparagraph}[1]{\oldsubparagraph{#1}\mbox{}}
\fi

%%% Use protect on footnotes to avoid problems with footnotes in titles
\let\rmarkdownfootnote\footnote%
\def\footnote{\protect\rmarkdownfootnote}

%%% Change title format to be more compact
\usepackage{titling}

% Create subtitle command for use in maketitle
\newcommand{\subtitle}[1]{
  \posttitle{
    \begin{center}\large#1\end{center}
    }
}

\setlength{\droptitle}{-2em}
  \title{Lewis \& Clark BLT cluster}
  \pretitle{\vspace{\droptitle}\centering\huge}
  \posttitle{\par}
  \author{Watzek DI, etc.}
  \preauthor{\centering\large\emph}
  \postauthor{\par}
  \predate{\centering\large\emph}
  \postdate{\par}
  \date{2018-02-19}

\usepackage{booktabs}

\begin{document}
\maketitle

{
\setcounter{tocdepth}{1}
\tableofcontents
}
\chapter*{About BLT}\label{about-blt}
\addcontentsline{toc}{chapter}{About BLT}

A description of BLT goes here.

\section{About the Cluster}\label{about-the-cluster}

content here\ldots{}.

\chapter{Getting Connected}\label{getting-connected}

\section{Accounts}\label{accounts}

In order to gain access to the cluster, you first need an account.
Contact the BLT Admins to request an account.

NOTE: Once you receive a temporary password, please reset it within 5
days of gaining access to the system.

\section{Getting on the network}\label{getting-on-the-network}

The BLT cluster is quite isolated from LC's public-facing
infrastructure. In order to connect to it, you will need a copy of Cisco
AnyConnect secure mobility client, which is available to LC students,
faculty, and staff HERE.

If you are using Linux to connect to the cluster, the current version of
Cisco AnyConnect will fail to install. Luckily, there is an open-source
equivalent called OpenConnect, which installs as a menu option for
debian and redhat based OSes. You will need to open your network
settings and click the green plus button to add a new connection, and
then select VPN when prompted. After that, put in
\texttt{vpn.lclark.edu} for the gateway option and the same root CA
certificate as you used when setting up LC secure. After you click save,
it will ask for your LC id and password.

After you have installed and started AnyConnect:

\begin{enumerate}
\def\labelenumi{\arabic{enumi}.}
\tightlist
\item
  Start a VPN session by typing \texttt{vpn.lclark.edu} in the text box
  and clicking ``connect''
\item
  When prompted, put in your LC username and password for access Now,
  your computer is connected to the same virtual network as the cluster.
\end{enumerate}

\section{Logging In}\label{logging-in}

NOTE: In order to log in, you will need an SSH client. If you are using
a Mac or Linux machine, you already have one. If you are using Windows,
you will need to install PuTTY or similar.

\begin{enumerate}
\def\labelenumi{\arabic{enumi}.}
\tightlist
\item
  Open your SSH client
\end{enumerate}

\begin{itemize}
\tightlist
\item
  On Mac Press the space bar and command key at the same time. then type
  ``Terminal'' and hit return
\item
  On Linux Open a terminal window
\item
  On Windows Open PuTTY
\end{itemize}

\begin{enumerate}
\def\labelenumi{\arabic{enumi}.}
\setcounter{enumi}{1}
\tightlist
\item
  Log In!
\end{enumerate}

\begin{itemize}
\tightlist
\item
  On Mac or Linux type
  \texttt{ssh\ \textless{}lclark\ username\textgreater{}@mayo.blt.lclark.edu}
  and type your password when prompted
\item
  On Windows Open PuTTY, set ``Host Name'' to mayo.blt.lclark.edu and
  click ``Open'', and follow the prompt. Congratulations! You have
  logged in to the BLT cluster! See Using the Cluster for more
  information about what you can do.
\end{itemize}

\chapter{Submitting Jobs}\label{submitting-jobs}

\section{Checking Usage}\label{checking-usage}

At any time, a user can check what the current availability of the
cluster is by typing \texttt{SGE\_Avail} on their command line. The
output will look something like this:

\begin{verbatim}
               #HOST  TOTRAM FREERAM    TOTSLOTS             Q  QSLOTS  QFREESLOTS   QSTATUS     QTYPE
               bacon   503.6   500.3          48         all.q      48          48    normal        BP
             lettuce   503.6   500.2          48         all.q      48          48    normal        BP
              tomato   503.6   500.2          48         all.q      48          48    normal        BP
\end{verbatim}

\chapter{Technical Details}\label{technical-details}

Now I'll teach you some crazy math, but I need to work it out
first\ldots{}

\bibliography{book.bib}


\end{document}
